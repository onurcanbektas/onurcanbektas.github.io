% Options for packages loaded elsewhere
\PassOptionsToPackage{unicode}{hyperref}
\PassOptionsToPackage{hyphens}{url}
\PassOptionsToPackage{dvipsnames,svgnames,x11names}{xcolor}
%
\documentclass[
  letterpaper,
  DIV=11,
  numbers=noendperiod]{scrartcl}

\usepackage{amsmath,amssymb}
\usepackage{iftex}
\ifPDFTeX
  \usepackage[T1]{fontenc}
  \usepackage[utf8]{inputenc}
  \usepackage{textcomp} % provide euro and other symbols
\else % if luatex or xetex
  \usepackage{unicode-math}
  \defaultfontfeatures{Scale=MatchLowercase}
  \defaultfontfeatures[\rmfamily]{Ligatures=TeX,Scale=1}
\fi
\usepackage{lmodern}
\ifPDFTeX\else  
    % xetex/luatex font selection
\fi
% Use upquote if available, for straight quotes in verbatim environments
\IfFileExists{upquote.sty}{\usepackage{upquote}}{}
\IfFileExists{microtype.sty}{% use microtype if available
  \usepackage[]{microtype}
  \UseMicrotypeSet[protrusion]{basicmath} % disable protrusion for tt fonts
}{}
\makeatletter
\@ifundefined{KOMAClassName}{% if non-KOMA class
  \IfFileExists{parskip.sty}{%
    \usepackage{parskip}
  }{% else
    \setlength{\parindent}{0pt}
    \setlength{\parskip}{6pt plus 2pt minus 1pt}}
}{% if KOMA class
  \KOMAoptions{parskip=half}}
\makeatother
\usepackage{xcolor}
\setlength{\emergencystretch}{3em} % prevent overfull lines
\setcounter{secnumdepth}{5}
% Make \paragraph and \subparagraph free-standing
\ifx\paragraph\undefined\else
  \let\oldparagraph\paragraph
  \renewcommand{\paragraph}[1]{\oldparagraph{#1}\mbox{}}
\fi
\ifx\subparagraph\undefined\else
  \let\oldsubparagraph\subparagraph
  \renewcommand{\subparagraph}[1]{\oldsubparagraph{#1}\mbox{}}
\fi

\usepackage{color}
\usepackage{fancyvrb}
\newcommand{\VerbBar}{|}
\newcommand{\VERB}{\Verb[commandchars=\\\{\}]}
\DefineVerbatimEnvironment{Highlighting}{Verbatim}{commandchars=\\\{\}}
% Add ',fontsize=\small' for more characters per line
\usepackage{framed}
\definecolor{shadecolor}{RGB}{241,243,245}
\newenvironment{Shaded}{\begin{snugshade}}{\end{snugshade}}
\newcommand{\AlertTok}[1]{\textcolor[rgb]{0.68,0.00,0.00}{#1}}
\newcommand{\AnnotationTok}[1]{\textcolor[rgb]{0.37,0.37,0.37}{#1}}
\newcommand{\AttributeTok}[1]{\textcolor[rgb]{0.40,0.45,0.13}{#1}}
\newcommand{\BaseNTok}[1]{\textcolor[rgb]{0.68,0.00,0.00}{#1}}
\newcommand{\BuiltInTok}[1]{\textcolor[rgb]{0.00,0.23,0.31}{#1}}
\newcommand{\CharTok}[1]{\textcolor[rgb]{0.13,0.47,0.30}{#1}}
\newcommand{\CommentTok}[1]{\textcolor[rgb]{0.37,0.37,0.37}{#1}}
\newcommand{\CommentVarTok}[1]{\textcolor[rgb]{0.37,0.37,0.37}{\textit{#1}}}
\newcommand{\ConstantTok}[1]{\textcolor[rgb]{0.56,0.35,0.01}{#1}}
\newcommand{\ControlFlowTok}[1]{\textcolor[rgb]{0.00,0.23,0.31}{#1}}
\newcommand{\DataTypeTok}[1]{\textcolor[rgb]{0.68,0.00,0.00}{#1}}
\newcommand{\DecValTok}[1]{\textcolor[rgb]{0.68,0.00,0.00}{#1}}
\newcommand{\DocumentationTok}[1]{\textcolor[rgb]{0.37,0.37,0.37}{\textit{#1}}}
\newcommand{\ErrorTok}[1]{\textcolor[rgb]{0.68,0.00,0.00}{#1}}
\newcommand{\ExtensionTok}[1]{\textcolor[rgb]{0.00,0.23,0.31}{#1}}
\newcommand{\FloatTok}[1]{\textcolor[rgb]{0.68,0.00,0.00}{#1}}
\newcommand{\FunctionTok}[1]{\textcolor[rgb]{0.28,0.35,0.67}{#1}}
\newcommand{\ImportTok}[1]{\textcolor[rgb]{0.00,0.46,0.62}{#1}}
\newcommand{\InformationTok}[1]{\textcolor[rgb]{0.37,0.37,0.37}{#1}}
\newcommand{\KeywordTok}[1]{\textcolor[rgb]{0.00,0.23,0.31}{#1}}
\newcommand{\NormalTok}[1]{\textcolor[rgb]{0.00,0.23,0.31}{#1}}
\newcommand{\OperatorTok}[1]{\textcolor[rgb]{0.37,0.37,0.37}{#1}}
\newcommand{\OtherTok}[1]{\textcolor[rgb]{0.00,0.23,0.31}{#1}}
\newcommand{\PreprocessorTok}[1]{\textcolor[rgb]{0.68,0.00,0.00}{#1}}
\newcommand{\RegionMarkerTok}[1]{\textcolor[rgb]{0.00,0.23,0.31}{#1}}
\newcommand{\SpecialCharTok}[1]{\textcolor[rgb]{0.37,0.37,0.37}{#1}}
\newcommand{\SpecialStringTok}[1]{\textcolor[rgb]{0.13,0.47,0.30}{#1}}
\newcommand{\StringTok}[1]{\textcolor[rgb]{0.13,0.47,0.30}{#1}}
\newcommand{\VariableTok}[1]{\textcolor[rgb]{0.07,0.07,0.07}{#1}}
\newcommand{\VerbatimStringTok}[1]{\textcolor[rgb]{0.13,0.47,0.30}{#1}}
\newcommand{\WarningTok}[1]{\textcolor[rgb]{0.37,0.37,0.37}{\textit{#1}}}

\providecommand{\tightlist}{%
  \setlength{\itemsep}{0pt}\setlength{\parskip}{0pt}}\usepackage{longtable,booktabs,array}
\usepackage{calc} % for calculating minipage widths
% Correct order of tables after \paragraph or \subparagraph
\usepackage{etoolbox}
\makeatletter
\patchcmd\longtable{\par}{\if@noskipsec\mbox{}\fi\par}{}{}
\makeatother
% Allow footnotes in longtable head/foot
\IfFileExists{footnotehyper.sty}{\usepackage{footnotehyper}}{\usepackage{footnote}}
\makesavenoteenv{longtable}
\usepackage{graphicx}
\makeatletter
\def\maxwidth{\ifdim\Gin@nat@width>\linewidth\linewidth\else\Gin@nat@width\fi}
\def\maxheight{\ifdim\Gin@nat@height>\textheight\textheight\else\Gin@nat@height\fi}
\makeatother
% Scale images if necessary, so that they will not overflow the page
% margins by default, and it is still possible to overwrite the defaults
% using explicit options in \includegraphics[width, height, ...]{}
\setkeys{Gin}{width=\maxwidth,height=\maxheight,keepaspectratio}
% Set default figure placement to htbp
\makeatletter
\def\fps@figure{htbp}
\makeatother

\KOMAoption{captions}{tableheading}
\makeatletter
\makeatother
\makeatletter
\makeatother
\makeatletter
\@ifpackageloaded{caption}{}{\usepackage{caption}}
\AtBeginDocument{%
\ifdefined\contentsname
  \renewcommand*\contentsname{Table of contents}
\else
  \newcommand\contentsname{Table of contents}
\fi
\ifdefined\listfigurename
  \renewcommand*\listfigurename{List of Figures}
\else
  \newcommand\listfigurename{List of Figures}
\fi
\ifdefined\listtablename
  \renewcommand*\listtablename{List of Tables}
\else
  \newcommand\listtablename{List of Tables}
\fi
\ifdefined\figurename
  \renewcommand*\figurename{Figure}
\else
  \newcommand\figurename{Figure}
\fi
\ifdefined\tablename
  \renewcommand*\tablename{Table}
\else
  \newcommand\tablename{Table}
\fi
}
\@ifpackageloaded{float}{}{\usepackage{float}}
\floatstyle{ruled}
\@ifundefined{c@chapter}{\newfloat{codelisting}{h}{lop}}{\newfloat{codelisting}{h}{lop}[chapter]}
\floatname{codelisting}{Listing}
\newcommand*\listoflistings{\listof{codelisting}{List of Listings}}
\makeatother
\makeatletter
\@ifpackageloaded{caption}{}{\usepackage{caption}}
\@ifpackageloaded{subcaption}{}{\usepackage{subcaption}}
\makeatother
\makeatletter
\@ifpackageloaded{tcolorbox}{}{\usepackage[skins,breakable]{tcolorbox}}
\makeatother
\makeatletter
\@ifundefined{shadecolor}{\definecolor{shadecolor}{rgb}{.97, .97, .97}}
\makeatother
\makeatletter
\makeatother
\makeatletter
\makeatother
\ifLuaTeX
  \usepackage{selnolig}  % disable illegal ligatures
\fi
\IfFileExists{bookmark.sty}{\usepackage{bookmark}}{\usepackage{hyperref}}
\IfFileExists{xurl.sty}{\usepackage{xurl}}{} % add URL line breaks if available
\urlstyle{same} % disable monospaced font for URLs
\hypersetup{
  pdftitle={How to interoperate between Singulariy and Docker in HPC environments},
  pdfauthor={Onurcan Bektas},
  colorlinks=true,
  linkcolor={blue},
  filecolor={Maroon},
  citecolor={Blue},
  urlcolor={Blue},
  pdfcreator={LaTeX via pandoc}}

\title{How to interoperate between Singulariy and Docker in HPC
environments}
\author{Onurcan Bektas}
\date{2024-12-01}

\begin{document}
\maketitle
\ifdefined\Shaded\renewenvironment{Shaded}{\begin{tcolorbox}[breakable, interior hidden, borderline west={3pt}{0pt}{shadecolor}, enhanced, boxrule=0pt, frame hidden, sharp corners]}{\end{tcolorbox}}\fi

\renewcommand*\contentsname{Table of contents}
{
\hypersetup{linkcolor=}
\setcounter{tocdepth}{3}
\tableofcontents
}
\hypertarget{abstract}{%
\section{Abstract}\label{abstract}}

The demand for computational power to carry out scientific research has
increased dramatically in the last decade. To deal with this demand,
high-performance computing (HPC) clusters has been established across
the world as a collaboration between research instituations and
universities. HPC clusters a bunch of inter-connected computers managed
by a workload manager, such as SLURM, where individual users can request
resources to carry out large-scale computations and these resources can
be optimally allocated by the workload manager to maximise the
utilisation of the available resources. Because these clusters are
shared by possibly tausends of users, only few administrators are
allowed to install software to these computers due to security reasons.
However, the limited permissions the individual users have might
restrict them from installing the necessary software that they might
need to carry out their research. Singularity and docker solves this
issue by allowing individual users to create and run virtual software
environmenments where they can install any software they like. In this
blog post, I'll show you how to create a docker container and use it
with singularity (a.k.a. apptainer).

\hypertarget{step-1-create-a-dockerfile}{%
\section{Step 1: Create a dockerfile}\label{step-1-create-a-dockerfile}}

To create a docker container, we first need to create what is called a
\emph{dockerfile}.

\begin{Shaded}
\begin{Highlighting}[]

\FunctionTok{touch}\NormalTok{ my.dockerfile}
\FunctionTok{vi}\NormalTok{ my.dockerfile}
\end{Highlighting}
\end{Shaded}

Inside the \texttt{my.dockerfile}, we need to list instructions for
Docker to create a docker container. The format a typically docker
container that I use looks as follows:

\begin{Shaded}
\begin{Highlighting}[]
\CommentTok{\# Dockerfile for Seurat 4.3.0}
\KeywordTok{FROM}\NormalTok{ rocker/r{-}ver:4.2.0}
\CommentTok{\# Install Seurat\textquotesingle{}s system dependencies}
\KeywordTok{RUN} \ExtensionTok{apt{-}get}\NormalTok{ update}
\KeywordTok{RUN} \ExtensionTok{apt{-}get}\NormalTok{ install }\AttributeTok{{-}y} \DataTypeTok{\textbackslash{}}
\NormalTok{    libhdf5{-}dev }\DataTypeTok{\textbackslash{}}
\NormalTok{    libcurl4{-}openssl{-}dev }\DataTypeTok{\textbackslash{}}
\NormalTok{    libssl{-}dev }\DataTypeTok{\textbackslash{}}
\NormalTok{    libpng{-}dev }\DataTypeTok{\textbackslash{}}
\NormalTok{    libboost{-}all{-}dev }\DataTypeTok{\textbackslash{}}
\NormalTok{    libxml2{-}dev }\DataTypeTok{\textbackslash{}}
\NormalTok{    openjdk{-}8{-}jdk }\DataTypeTok{\textbackslash{}}
\NormalTok{    python3{-}dev }\DataTypeTok{\textbackslash{}}
\NormalTok{    python3{-}pip }\DataTypeTok{\textbackslash{}}
\NormalTok{    wget }\DataTypeTok{\textbackslash{}}
\NormalTok{    git }\DataTypeTok{\textbackslash{}}
\NormalTok{    libfftw3{-}dev }\DataTypeTok{\textbackslash{}}
\NormalTok{    libgsl{-}dev }\DataTypeTok{\textbackslash{}}
\NormalTok{    pkg{-}config}
\KeywordTok{RUN} \ExtensionTok{apt{-}get}\NormalTok{ install }\AttributeTok{{-}y}\NormalTok{ llvm{-}10}
\CommentTok{\# Install system library for rgeos}
\KeywordTok{RUN} \ExtensionTok{apt{-}get}\NormalTok{ install }\AttributeTok{{-}y}\NormalTok{ libgeos{-}dev}
\CommentTok{\# Install UMAP}
\KeywordTok{RUN} \VariableTok{LLVM\_CONFIG}\OperatorTok{=}\NormalTok{/usr/lib/llvm{-}10/bin/llvm{-}config }\ExtensionTok{pip3}\NormalTok{ install llvmlite}
\KeywordTok{RUN} \ExtensionTok{pip3}\NormalTok{ install numpy}
\KeywordTok{RUN} \ExtensionTok{pip3}\NormalTok{ install umap{-}learn}
\KeywordTok{RUN} \FunctionTok{git}\NormalTok{ clone }\AttributeTok{{-}{-}branch}\NormalTok{ v1.2.1 https://github.com/KlugerLab/FIt{-}SNE.git}
\KeywordTok{RUN} \ExtensionTok{g++} \AttributeTok{{-}std}\OperatorTok{=}\NormalTok{c++11 }\AttributeTok{{-}O3}\NormalTok{ FIt{-}SNE/src/sptree.cpp FIt{-}SNE/src/tsne.cpp FIt{-}SNE/src/nbodyfft.cpp  }\AttributeTok{{-}o}\NormalTok{ bin/fast\_tsne }\AttributeTok{{-}pthread} \AttributeTok{{-}lfftw3} \AttributeTok{{-}lm}
\CommentTok{\# Install bioconductor dependencies \& suggests}
\KeywordTok{RUN} \ExtensionTok{R} \AttributeTok{{-}{-}no{-}echo} \AttributeTok{{-}{-}no{-}restore} \AttributeTok{{-}{-}no{-}save} \AttributeTok{{-}e} \StringTok{"install.packages(\textquotesingle{}BiocManager\textquotesingle{})"}
\end{Highlighting}
\end{Shaded}

In each line, the capitalised words are instructions for docker to tell
what to do with the code following those words. For example, \emph{FROM}
tells docker that it should use the docker container located at
\href{https://hub.docker.com/layers/rocker/r-ver/4.2.0/images/sha256-53e27eaf50320246234dc960c19143161797b721b67d69f4fd4c2a5157b23b54?context=explore}{hub.docker.com/layers/rocker/r-ver/4.2.0}
as a \emph{base} image and build the container on top of that image.
This is typically used to set the operating system of the docker
container, e.g.~ubuntu, fedora, etc.. In this particular case, I'm using
official docker container of \textbf{R} programming language version
4.2.0 as the base so that I don't have to install I myself.

The lines folling the first line tells which commands to execute to set
up the container. For example, in the second line, I'm first updating
the virtual operating system of the docker container, then in the third
line I'm installing some libraries to the operating system that will be
needed for the R packages that I use. The details of how to set up a
docker container is beyond the scope of this blog post, so I'll skip
that for the moment.

Once you finish setting up the dockerfile, now we are ready to actually
build the container and \emph{upload} to somewhere so that we can access
it whenever we want.

\hypertarget{step-2-create-the-container-and-push-it-to-dockerhub}{%
\section{Step 2: Create the container and push it to
dockerhub}\label{step-2-create-the-container-and-push-it-to-dockerhub}}

Assuming that you are creating the docker container in a non-linux
environment, we will tell docker that the container will be used in the
a linux environment while we create the container.

\begin{Shaded}
\begin{Highlighting}[]
\ExtensionTok{docker}\NormalTok{ build }\AttributeTok{{-}t}\NormalTok{ mycontainer:latest . }\AttributeTok{{-}f}\NormalTok{ my.dockerfile  }\AttributeTok{{-}{-}platform}\NormalTok{ linux/x86\_64 }\DecValTok{2}\OperatorTok{\textgreater{}\&}\DecValTok{1} \KeywordTok{|} \FunctionTok{tee}\NormalTok{ build.log}
\end{Highlighting}
\end{Shaded}

Here, \texttt{mycontainer:latest} gives the container a name and a tag
to easily identify it. The tag is like a version of that container. The
dot \texttt{.} following that tells docker that the docker file
\texttt{-f\ my.dockerfile} is located at the current location where this
command is executed. The parameter \texttt{-\/-platform\ linux/x86\_64}
tells docker that the container should be compatible with a linux host.
Once you execute the above command, it can take from minutes to hours to
create the container, depending on what you asked the docker to install
to your container.

Once the creation of the container is complete, we need to push it a
dockerhub as a repository. First create an account at
\href{https://hub.docker.com/}{dockerhub}. Then create a reposity at
dockerhub in the website. You will push the container you created in
your computer to this repo you created in dockerhub. Then from your
terminal, log into your account

\begin{Shaded}
\begin{Highlighting}[]
\ExtensionTok{docker}\NormalTok{ login }\AttributeTok{{-}{-}username} \OperatorTok{\textless{}}\NormalTok{dockerhub username}\OperatorTok{\textgreater{}}
\end{Highlighting}
\end{Shaded}

Now you first need to tag the docker container in your computer to the
repo in the dockerhub. List all docker images and copy
\texttt{IMAGE\ ID} of that container. It should be a 12 character
alphanumeric string, something like \texttt{76ad0cae35c3}.

\begin{Shaded}
\begin{Highlighting}[]
\ExtensionTok{docker}\NormalTok{ image ls}
\end{Highlighting}
\end{Shaded}

Once you find the \texttt{IMAGE\ ID} of your container, we tag it to the
repo as

\begin{Shaded}
\begin{Highlighting}[]
\ExtensionTok{docker}\NormalTok{ tag }\OperatorTok{\textless{}}\NormalTok{IMAGE ID}\OperatorTok{\textgreater{}} \OperatorTok{\textless{}}\NormalTok{dockerhub username}\OperatorTok{\textgreater{}}\NormalTok{/}\OperatorTok{\textless{}}\NormalTok{repo name}\OperatorTok{\textgreater{}}
\end{Highlighting}
\end{Shaded}

Now we are ready push the container that is still stored in your
computer to the repo at
\texttt{\textless{}dockerhub\ username\textgreater{}/\textless{}repo\ name\textgreater{}}.

\begin{Shaded}
\begin{Highlighting}[]
\ExtensionTok{docker}\NormalTok{ push }\OperatorTok{\textless{}}\NormalTok{dockerhub username}\OperatorTok{\textgreater{}}\NormalTok{/}\OperatorTok{\textless{}}\NormalTok{repo name}\OperatorTok{\textgreater{}}
\end{Highlighting}
\end{Shaded}

It might take some time for the push to finish, if this is your first
time creating this container and it is a large container.

\hypertarget{step-3-use-the-docker-container-with-singulariyapptainer}{%
\section{Step 3: Use the docker container with
Singulariy/Apptainer}\label{step-3-use-the-docker-container-with-singulariyapptainer}}

Having created the docker container and stored in the dockerhub, we are
ready to use with singularity. Log into the HPC cluster and make sure
that the singularity/apptainer is installed already. We first make sure
that the cache directories are created - otherwise singularity will
complain. Then we can directly run the docker container with singularity
in the following way:

\begin{Shaded}
\begin{Highlighting}[]
\NormalTok{mkdir }\SpecialCharTok{$}\NormalTok{SINGULARITY\_TMPDIR }
\NormalTok{mkdir }\SpecialCharTok{$}\NormalTok{APPTAINER\_CACHEDIR}
\NormalTok{singularity run  docker}\SpecialCharTok{:}\ErrorTok{//\textless{}}\NormalTok{dockerhub username}\SpecialCharTok{\textgreater{}}\ErrorTok{/\textless{}}\NormalTok{repo name}\SpecialCharTok{\textgreater{}}\ErrorTok{:}\NormalTok{latest}
\end{Highlighting}
\end{Shaded}

Here, the word \texttt{docker} before
\texttt{\textless{}dockerhub\ username\textgreater{}/\textless{}repo\ name\textgreater{}:latest}
tells singularity to search for the repo in the dockerhub. With this,
singularity download the docker container at
\texttt{\textless{}dockerhub\ username\textgreater{}/\textless{}repo\ name\textgreater{}:latest}
and run it. Once it is done, you will be inside the container and start
using the software that you have installed in that container.

However, typically, when we are running the docker container, we also
would like to access to the data is that in the host environment. As a
result, we need to \emph{bind} the some of the directories in the host
to the a directory in the virtual operating system. \emph{Binding}
simply means that you make a directory in the host operating system
accessible from within the container. To accmplish this we pass
\texttt{-\/-bind} parameter. The path before \texttt{:} indicates
directory in the host, and the path after \texttt{:} indicates the path
inside the container.

\begin{Shaded}
\begin{Highlighting}[]
\NormalTok{singularity run }\SpecialCharTok{{-}{-}}\NormalTok{bind }\SpecialCharTok{\textless{}}\NormalTok{host path}\SpecialCharTok{\textgreater{}}\ErrorTok{:\textless{}}\NormalTok{container path}\SpecialCharTok{\textgreater{}}\NormalTok{ docker}\SpecialCharTok{:}\ErrorTok{//\textless{}}\NormalTok{dockerhub username}\SpecialCharTok{\textgreater{}}\ErrorTok{/\textless{}}\NormalTok{repo name}\SpecialCharTok{\textgreater{}}\ErrorTok{:}\NormalTok{latest}
\end{Highlighting}
\end{Shaded}




\end{document}
